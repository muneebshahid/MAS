\documentclass[a4paper,12pt]{article}
\usepackage[utf8]{inputenc}
\usepackage{amsmath}
%opening
\title{Advanced Software Technology}
\author{Muneeb Shahid}

\begin{document}

\begin{titlepage}
\pagenumbering{gobble} 
\maketitle

\end{titlepage}

\subsection*{• The Vacuum Cleaning Robot Problem}
\pagenumbering{arabic} 
\subsubsection*{How can you mathematically describe the performance of each vacuum cleaning robot?}
The first robot cleans .5A, where A is the whole area. 
The second robot's function is constant, since no matter what it will always clean .7A of the room and make the rest dirty.
The third robot cleans .3A and makes .1A dirty.

\subsubsection*{How can you mathematically formulate a strategy for using the vacuum cleaning robots?}
This requires using sum of all permutations. Using permutations we can find out the possible strategies and then 
execute them to find out which one is better. 

\subsubsection*{How many different strategies exist for the problem on hand?}
16 in total. 
\newline 6 if all 3 robots are selected. 
\newline 6 if 2 out of 3 are selected.
\newline 3 if 1 out of 3 are selected. 
\newline 1 if no robot is selected.

\subsubsection*{How can you compare two strategies and decide which one is better?}
This can be found out just by checking the overall cleanliness of the room.

\subsubsection*{Can you prove that for any two strategies, there will always be one better than another?}
Different strategies are better for different scenarios. No strategy will always be better.

\subsection*{• VCR Problem Extension}
\subsubsection*{Given the same constraints for strategies as in the previous exercise, and a number n of
available vacuum cleaning robots, derive a formula that computes the number of possible
strategies. Explain how you derived the formula.}
What we need here is a sum of all permutations, Sigma nPr. 
3P3 give all possibilities if 3 out of 3 robots are allowed to choose from,
3P2 for 2 out of 3 robots and so on. The sum of all these permutations gives us the required all possibilities.

\subsubsection*{If we assume that the performance of each vacuum cleaning robot can be described by a
linear function, how can you then describe a strategy?}
This is the function that I came up with:
\newline
{\tt cleanRoomBy * totalArea + (1 - (cleanRoomBy + dirtyRoomBy)) * percentCleanArea}

Lets say a robot cleans by .5(\textit{totalArea}) and makes a room dirty by .1(\textit{totalArea}).
Multiplying \textit{cleanRoomBy}(.5) with \textit{totalArea} gives the area that is definitely cleaned by the robot.
Subtracting the sum of \textit{cleanRoomBy}(.5) and \textit{dirtyRoomBy}(.1) from 1 and then multiplying by \textit{percentCleanArea} gives the area 
which was already clean. Adding these values gives the total cleaned area.
\subsection*{• Confused Brick Layer Problem}
\subsubsection*{Could not complete the solution.}

\subsection*{• Perception Seminar Problem}
\subsubsection*{As usual, develop first a suitable formal description of the problem}
A logic based problem, used contradictions to figure out the problem. We can use combination 12C6 to be precise for this problem.
\subsubsection*{Develop a formal method that can determine whether a suggested solution actually is one or not}
We can first start with one combination assuming people in this combination are not blindfolded and check if the set passes saftey checks.
If one person on this list says that another person who is not in the list is suppose to be on the list then obviously this solution is not correct.
There is another special condition that can be applied earlier to reduce the search space, say Gonzo says is Hosagi is not blindfolded and
Hosagi says Gonzo is not blindfolded. Then they both are not blindfolded for sure. 
\newline
  Hosagi	|	Gonzo\newline
  B		|	B\newline
  B		|	N\newline
  N		|	B\newline
  N		|	N\newline
 
 \subsubsection*{Assuming you would systematically search for a solution, what would be the size of the search space?}
 This is nCk in general and for this solution it is 12C6 = 924
\end{document}